\documentclass[12pt]{article}

\usepackage{amsmath,amssymb}

\newcommand{\Z}{\mathbb{Z}}

\begin{document}

\begin{center}
{\bf Math 5125}\\
HW 1
\smallskip

Due: Wednesday, September 1
\end{center}


\begin{enumerate}

\item Suppose $A \overset{f}{\to} B$ and $C\subseteq A$.
\begin{enumerate}
\item Prove that $C\subseteq f^{-1}(f(C))$.
\item Prove that $C=f^{-1}(f(C))$ if $f$ is injective.
\end{enumerate}

\smallskip
{\bf Proof:} (a) For $c \in C$, by the definition of image, we have $f(c) \in f(C)$, which is equivalent to $c \in f^{-1}(f(C))$, by the definition of pre-image. Therefore, $c \in C \Rightarrow c \in f^{-1}(f(C))$, i.e. $C \subseteq f^{-1}(f(C))$.

(b) All we need to show is that if $f$ is injective, $f^{-1}(f(C)) \subseteq C$. For $c \in f^{-1}(f(C))$, we have $f(c) \in f(C)$. This means that there exists $c_0 \in C$, such that $f(c) = f(c_0)$. And $f$ is injective means that $f(c) = f(c_0) \Rightarrow c = c_0$. Therefore, $c \in C$. We have $f^{-1}(f(c)) \subseteq C$ and this together with (a) completes the proof.

\newpage

\item Let $a=12$ and $b=17$. Use the Euclidean algorithm to find $d=\gcd(a,b)$ and express $d$ as a linear combination of $a$ and $b$. Does $a$ have a multiplicative inverse modulo $b$? If so, what is $\overline{a}^{-1}\in\Z/b\Z$?

\smallskip
{\bf Proof:} Using the Euclidean algorithm,
\begin{align}
    17 = 12 \times 1 + 5\cr
    12 = 5 \times 2 + 2\cr
    5 = 2 \times 2 + 1\nonumber
\end{align}
Therefore, for $a = 12$ and $b = 17$, $d = \gcd(a,b) = 1$. To express $d$ as a linear combination of $a$ and $b$, we have
\begin{align}
    d = 1 &= 5 - 2 \times 2 \cr
     &= 5 - (12 - 5 \times 2) \times 2 \cr
     &= 5 \times 5 - 12 \times 2 \cr
     &= (17 - 12 \times 1) \times 5 - 12 \times 2 \cr
     &= 17 \times 5 - 12 \times 7 \cr
     &= -7 a + 5 b~. \nonumber
\end{align}
Since $1 = -7 a + 5 b$, applying modulo $b$ to this equation, we have
$$
1 \equiv -7 a \mod b~.
$$
Therefore,
$$
\overline{a}^{-1} = \overline{-7} = \overline{10} \in \Z/b\Z~.
$$

\newpage
\item Let $n$ be a positive integer. For any $\overline{a}\in\Z/n\Z$ (where $a\in\Z$), define the set
\begin{align*}
( \overline{a} ) &= \{\overline{ka} \mid k \in \mathbb{Z}\} \subseteq \Z/n\Z.
\end{align*}
Prove that $( \overline{a} ) = ( \overline{d} )$ where $d=\gcd(a,n)$.

\smallskip
{\bf Proof:} For given $a, n \in \Z$, we write their greatest common divisor $d = \gcd(a,n)$.
%Since we are only considering equivalence classes, without losing of generality, suppose $0 \le a < n$.
Write $a = r d$, where $r \in \Z$.

If $\overline{x} \in (\overline{a})$, then there exists $m ,k \in \Z$:
$$
x = m n + k a = m n + (k r) d~,
$$
therefore $\overline{x} \in (\overline{d})$, so $(\overline{a}) \subset (\overline{d})$.

If $\overline{x} \in (\overline{d})$, then there exists $m, k \in \Z$:
$$
x = mn + k d~.
$$
Also, $d$ can be written as a linear combination of $a$ and $n$,
$$
d = p a + q n \quad p,q \in \Z~.
$$
Therefore,
$$
x = mn + k(pa + qn) = (m + kq) n + (kp) a~,
$$
so $\overline{x} \in (\overline{a})$ and $(\overline{d}) \subset (\overline{a})$. Finally, $(\overline{a}) = (\overline{d})$ where $d = \gcd(a,n)$.

\newpage

\item Show that the set
\begin{align*}
S=\Big\{\begin{pmatrix}
a & -b \\
b & a
\end{pmatrix} {\Big|}\ a,b\in \mathbb{R}\Big\}.
\end{align*}
is {\em closed} under matrix addition and multiplication, i.e., $A,B\in S$ implies $A+B\in S$ and $AB\in S$. (Do the formulas for these operations look familiar?)

\smallskip
{\bf Proof:} For any $A, B \in S$, say
$$
A = \begin{pmatrix}
a_1 & -b_1 \\
b_1 & a_1
\end{pmatrix}~, \quad
B = \begin{pmatrix}
a_2 & -b_2 \\
b_2 & a_2
\end{pmatrix}~,\quad
a_i, b_i \in \mathbb{R}~.
$$
Since $\mathbb{R}$ is closed under addition and multiplication. Then
$$
A + B = \begin{pmatrix}
a_1 + a_2 & - (b_1 + b_2) \\
b_1 + b_2 & a_1 + a_2
\end{pmatrix} \in S~,
$$
and
$$
A B = \begin{pmatrix}
    a_1 a_2 - b_1 b_2 & - (a_1 b_2 + a_2 b_1) \\
    a_1 b_2 + a_2 b_1 & a_1 a_2 - b_1 b_2
\end{pmatrix} \in S~.
$$
Therefore, the matrix addition and multiplication over the set $S =\Big\{ \begin{pmatrix}
    a & -b \\
    b & a
    \end{pmatrix} {\Big |} \  a,b \in \mathbb{R} \Big\}$ looks familiar with addition and multiplication over complex numbers $\mathbb{C} = \{ a + b i \ | \ a, b\in \mathbb{R} \}$.

\end{enumerate}

\end{document}